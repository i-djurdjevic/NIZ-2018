%!TeX program = pdflatex
%!BIB program = biber

\documentclass[serbian]{beamer}

\newcommand\tab[1][1cm]{\hspace*{#1}}
\usepackage{babel}
\mode<presentation>
{
  \usetheme{default}      % or try Boadilla, Rochester, ...
  \usecolortheme{default} % or try beaver, dove, ...
  \usefonttheme{default}  % or try serif, structurebold, ...
  \setbeamertemplate{navigation symbols}{}
  \setbeamertemplate{caption}[numbered]
} 

\usepackage[style=numeric]{biblatex}
\addbibresource{serbian.bib}
\setbeamertemplate{bibliography item}[triangle]

% This is for slide numbering
\setbeamertemplate{sidebar right}{}
\setbeamertemplate{footline}{%
	\hfill\usebeamertemplate***{navigation symbols}
	\large\insertframenumber{}/\inserttotalframenumber\hspace{0.5cm}\vspace{0.2cm}
}

\usepackage{listings}
\lstset{
	basicstyle=\ttfamily\small,
	columns=fullflexible,
	keepspaces=true,
	showstringspaces=false,
	escapechar=\%,
	tabsize=4
}

\usepackage{amsmath}
\newcommand\numberthis{\addtocounter{equation}{1}\tag{\theequation}}

\title[Jakobijev algoritam za pronalaženje sopstvenih vrednosti]{Jakobijev algoritam za pronalaženje sopstvenih vrednosti}
\author{Isidora Đurđević}
\institute{Matematički fakultet, \\ Univerzitet u Beogradu}
\date{\today.}

\begin{document}

\begin{frame}
  	\titlepage
\end{frame}

\begin{frame}{Sadr\v zaj}
  	\tableofcontents
\end{frame}

\section{Uvod}

\begin{frame}{Uvod}

\begin{itemize}
	\item Jakobijev algoritam je iterativna metoda za pronalaženje sopstvenih vrednosti i sopstvenih vektora matrice
	\item Naziv je dobio po matematičaru Gustav Karl Jakobiju
	\item Poznat kao proces dijagonalizacije
\end{itemize}

\end{frame}

\section{Jakobijev algoritam}

\begin{frame}{Jakobijev algoritam}

\begin{itemize}
	\item Rešenje je garantovano za realne i simetrične matrice
	\item Zasniva se na iterativnoj primeni matrica rotacija kako bi se anulirali elementi van dijagonale
	\item Rotacije se zovu Jakobijeve ili Givensove rotacije
	\item Rezultat je dijagonalna matrica na čijoj dijagonali se nalaze sopstvene vrednosti
	\item Proizvodom primenjenih matrica rotacija dobija se matrica čije kolone čine sopstvene vektore
\end{itemize}

\end{frame}

\begin{frame}{Jakobijev algoritam (2)}

\begin{itemize}
	\item Neka je \texttt{S} simetrična matrica, i \texttt{G = G(i, j, $\theta$)} Givensova matrica rotacije. Tada je:
\end{itemize}

	 \begin{center}
 		\texttt{S' = GS$G^{T}$} \\
 	\end{center}
	
\begin{itemize}
	\item simetrična matrica i slična je matrici \texttt{S}
	\item matrica \texttt{S'} se izračunava sledećim formulama:
\end{itemize}

	\begin{center}
 		\texttt{$S'_{ii}$ = $c^2S_{ii} - 2scS_{ij} + s^2S_{jj}$} \\
		\texttt{$S'_{jj}$ = $s^2S_{ii} - 2scS_{ij} + c^2S_{jj}$} \\
		\texttt{$S'_{ij}$ = $S'_{ji}$ = $(c^2 - s^2)S_{ij} + sc(S_{ii} - S_{jj})$} \\
		\texttt{$S'_{ik}$ = $S'_{ki}$ = $cS_{ik} - sS_{jk}$} \tab[1cm] k $\neq$ i, j \\
		\texttt{$S'_{jk}$ = $S'_{kj}$ = $sS_{ik} - cS_{jk}$} \tab[1cm] k $\neq$ i, j  \\
		\texttt{$S'_{kl}$ = $S_{kl}$} \tab[3cm] k, l $\neq$ i, j  
 	\end{center}

\end{frame}

\begin{frame}{Jakobijev algoritam (3)}

\begin{itemize}
	\item Gde je \texttt{s = sin($\theta$)} i \texttt{c = cos($\theta$)}
	\item Pošto je \texttt{G} ortogonalna matrica, \texttt{S} i \texttt{S'} imaju istu normu
	\item Ako izaberemo $\theta$ tako da je \texttt{$S'_{ij}$ = 0} tada je \texttt{S'}:
\end{itemize}

	\begin{equation}
		S'_{ij} = cos(2\theta)S_{ij} + \dfrac{1}{2}sin(2\theta)(S_{ii} - S_{jj})
	\end{equation}

\end{frame}

\begin{frame}{Jakobijev algoritam (4)}

	\begin{itemize}
		\item Izjednačavanjem prethodne formule na 0 i sređivanjem dobijamo:
	\end{itemize}
	
	\begin{equation}
		tan(2\theta) = \dfrac{2S_{ij}}{S_{jj} - S_{ii}}
	\end{equation}
	
	\begin{itemize}
		\item $S_{ij}$ je pivot i predstavlja najveći vandijagonalni element
		\item Matrica rotacije:
	\end{itemize}
	
	\[\begin{bmatrix}
    			cos\theta & sin\theta  \\
   			-sin\theta   &  cos\theta  \\
	\end{bmatrix}\]
	
\end{frame}

\section{Jakobijev algoritam - koraci}

\begin{frame}{Jakobijev algoritam - koraci}

\begin{itemize}
	\item Pronađi najveći vandijagonalni element
	\item Ukoliko je najveći vandijagonalni element manji od unapred definisane vrednosti $\varepsilon$ - \texttt{break} -
	\item Inače izvrši rotaciju tako da se anulira najveći vandijagonalni element
\end{itemize}

\end{frame}

\begin{frame}{Ocena slo\v zenosti}

\begin{itemize}
	\item Složenost rotacije je \texttt{O(n)} ukoliko je poznat najveći vandijagonalni element, inače \texttt{O($n^2$)}
	\item Moguće je svesti na \texttt{O(n)} ukoliko se čuva niz sa vrednostima vandijagonalnih elemenata, gde je \texttt{i}-ti element niza najveći element u \texttt{i}-tom redu matrice
	\item Rotacija predstavlja množenje matrica i složenost je \texttt{O($n^3$)}
\end{itemize}

\end{frame}

\begin{frame}

\vspace*{3em}

{\Large Hvala na pa\v znji!} 

\vspace*{1em}

Pitanja?

\end{frame}

\section{Dodatak}

\begin{frame}{Dodatak}

\begin{flushleft}
	{\Large Literatura}
\end{flushleft}

\nocite{NumOpt}

\printbibliography

\begin{flushleft}
	{\Large Repozitorijum}
\end{flushleft}

\href{https://github.com/i-djurdjevic/NIZ-2018}{https://github.com/i-djurdjevic/NIZ-2018}

\end{frame}

\end{document}
